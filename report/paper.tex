\documentclass[journal, a4paper]{IEEEtran}

% some very useful LaTeX packages include:

%\usepackage{cite}      % Written by Donald Arseneau
                        % V1.6 and later of IEEEtran pre-defines the format
                        % of the cite.sty package \cite{} output to follow
                        % that of IEEE. Loading the cite package will
                        % result in citation numbers being automatically
                        % sorted and properly "ranged". i.e.,
                        % [1], [9], [2], [7], [5], [6]
                        % (without using cite.sty)
                        % will become:
                        % [1], [2], [5]--[7], [9] (using cite.sty)
                        % cite.sty's \cite will automatically add leading
                        % space, if needed. Use cite.sty's noadjust option
                        % (cite.sty V3.8 and later) if you want to turn this
                        % off. cite.sty is already installed on most LaTeX
                        % systems. The latest version can be obtained at:
                        % http://www.ctan.org/tex-archive/macros/latex/contrib/supported/cite/

\usepackage{graphicx}   % Written by David Carlisle and Sebastian Rahtz
                        % Required if you want graphics, photos, etc.
                        % graphicx.sty is already installed on most LaTeX
                        % systems. The latest version and documentation can
                        % be obtained at:
                        % http://www.ctan.org/tex-archive/macros/latex/required/graphics/
                        % Another good source of documentation is "Using
                        % Imported Graphics in LaTeX2e" by Keith Reckdahl
                        % which can be found as esplatex.ps and epslatex.pdf
                        % at: http://www.ctan.org/tex-archive/info/

%\usepackage{psfrag}    % Written by Craig Barratt, Michael C. Grant,
                        % and David Carlisle
                        % This package allows you to substitute LaTeX
                        % commands for text in imported EPS graphic files.
                        % In this way, LaTeX symbols can be placed into
                        % graphics that have been generated by other
                        % applications. You must use latex->dvips->ps2pdf
                        % workflow (not direct pdf output from pdflatex) if
                        % you wish to use this capability because it works
                        % via some PostScript tricks. Alternatively, the
                        % graphics could be processed as separate files via
                        % psfrag and dvips, then converted to PDF for
                        % inclusion in the main file which uses pdflatex.
                        % Docs are in "The PSfrag System" by Michael C. Grant
                        % and David Carlisle. There is also some information
                        % about using psfrag in "Using Imported Graphics in
                        % LaTeX2e" by Keith Reckdahl which documents the
                        % graphicx package (see above). The psfrag package
                        % and documentation can be obtained at:
                        % http://www.ctan.org/tex-archive/macros/latex/contrib/supported/psfrag/

%\usepackage{subfigure} % Written by Steven Douglas Cochran
                        % This package makes it easy to put subfigures
                        % in your figures. i.e., "figure 1a and 1b"
                        % Docs are in "Using Imported Graphics in LaTeX2e"
                        % by Keith Reckdahl which also documents the graphicx
                        % package (see above). subfigure.sty is already
                        % installed on most LaTeX systems. The latest version
                        % and documentation can be obtained at:
                        % http://www.ctan.org/tex-archive/macros/latex/contrib/supported/subfigure/

\usepackage{url}        % Written by Donald Arseneau
                        % Provides better support for handling and breaking
                        % URLs. url.sty is already installed on most LaTeX
                        % systems. The latest version can be obtained at:
                        % http://www.ctan.org/tex-archive/macros/latex/contrib/other/misc/
                        % Read the url.sty source comments for usage information.

%\usepackage{stfloats}  % Written by Sigitas Tolusis
                        % Gives LaTeX2e the ability to do double column
                        % floats at the bottom of the page as well as the top.
                        % (e.g., "\begin{figure*}[!b]" is not normally
                        % possible in LaTeX2e). This is an invasive package
                        % which rewrites many portions of the LaTeX2e output
                        % routines. It may not work with other packages that
                        % modify the LaTeX2e output routine and/or with other
                        % versions of LaTeX. The latest version and
                        % documentation can be obtained at:
                        % http://www.ctan.org/tex-archive/macros/latex/contrib/supported/sttools/
                        % Documentation is contained in the stfloats.sty
                        % comments as well as in the presfull.pdf file.
                        % Do not use the stfloats baselinefloat ability as
                        % IEEE does not allow \baselineskip to stretch.
                        % Authors submitting work to the IEEE should note
                        % that IEEE rarely uses double column equations and
                        % that authors should try to avoid such use.
                        % Do not be tempted to use the cuted.sty or
                        % midfloat.sty package (by the same author) as IEEE
                        % does not format its papers in such ways.

\usepackage{amsmath}    % From the American Mathematical Society
                        % A popular package that provides many helpful commands
                        % for dealing with mathematics. Note that the AMSmath
                        % package sets \interdisplaylinepenalty to 10000 thus
                        % preventing page breaks from occurring within multiline
                        % equations. Use:
%\interdisplaylinepenalty=2500
                        % after loading amsmath to restore such page breaks
                        % as IEEEtran.cls normally does. amsmath.sty is already
                        % installed on most LaTeX systems. The latest version
                        % and documentation can be obtained at:
                        % http://www.ctan.org/tex-archive/macros/latex/required/amslatex/math/
\usepackage{enumerate}
\usepackage{algorithm,algorithmic}

% Other popular packages for formatting tables and equations include:

%\usepackage{array}
% Frank Mittelbach's and David Carlisle's array.sty which improves the
% LaTeX2e array and tabular environments to provide better appearances and
% additional user controls. array.sty is already installed on most systems.
% The latest version and documentation can be obtained at:
% http://www.ctan.org/tex-archive/macros/latex/required/tools/

% V1.6 of IEEEtran contains the IEEEeqnarray family of commands that can
% be used to generate multiline equations as well as matrices, tables, etc.

% Also of notable interest:
% Scott Pakin's eqparbox package for creating (automatically sized) equal
% width boxes. Available:
% http://www.ctan.org/tex-archive/macros/latex/contrib/supported/eqparbox/

% *** Do not adjust lengths that control margins, column widths, etc. ***
% *** Do not use packages that alter fonts (such as pslatex).         ***
% There should be no need to do such things with IEEEtran.cls V1.6 and later.


% Your document starts here!
\begin{document}
\begin{titlepage}

\newcommand{\HRule}{\rule{\linewidth}{0.5mm}} % Defines a new command for the horizontal lines, change thickness here

\center % Center everything on the page
 %----------------------------------------------------------------------------------------
%	LOGO SECTION
%----------------------------------------------------------------------------------------

~\\[1cm]
\includegraphics{SCUT.png}\\[2cm] % Include a department/university logo - this will require the graphicx package

%----------------------------------------------------------------------------------------
%	TITLE SECTION
%----------------------------------------------------------------------------------------

\HRule \\[1cm]
{ \huge \bfseries The Experiment Report of \textit{Machine Learning} }\\[0.6cm] % Title of your document
\HRule \\[2cm]
%----------------------------------------------------------------------------------------
%	HEADING SECTIONS
%----------------------------------------------------------------------------------------


\textsc{\LARGE \textbf{School:} School of Software Engineering}\\[1cm]
\textsc{\LARGE \textbf{Subject:} Software Engineering}\\[2cm] 

 
%----------------------------------------------------------------------------------------
%	AUTHOR SECTION
%----------------------------------------------------------------------------------------

\begin{minipage}{0.4\textwidth}
\begin{flushleft} \large
\emph{Author:}\\
Jining He and Hui Han % Your name
\end{flushleft}
\end{minipage}
~
\begin{minipage}{0.4\textwidth}
\begin{flushright} \large
\emph{Supervisor:} \\
Qingyao Wu % Supervisor's Name
\end{flushright}
\end{minipage}\\[2cm]
~
\begin{minipage}{0.4\textwidth}
\begin{flushleft} \large
\emph{Student ID:}\\
201721045565 and 201721045572
\end{flushleft}
\end{minipage}
~
\begin{minipage}{0.4\textwidth}
\begin{flushright} \large
\emph{Grade:} \\
Graduate
\end{flushright}
\end{minipage}\\[2cm]

% If you don't want a supervisor, uncomment the two lines below and remove the section above
%\Large \emph{Author:}\\
%John \textsc{Smith}\\[3cm] % Your name

%----------------------------------------------------------------------------------------
%	DATE SECTION
%----------------------------------------------------------------------------------------

{\large \today}\\[2cm] % Date, change the \today to a set date if you want to be precise

 
%----------------------------------------------------------------------------------------

\vfill % Fill the rest of the page with whitespace

\end{titlepage}

% Define document title and author
	\title{Recommender System Based on Matrix Factorization}
	\maketitle

% Write abstract here
\begin{abstract}
    In this experiment, we build a recommender system based on matrix factorization. We train the model on MovieLens-100k dataset. The result shows that Matrix Factorization is a useful model-based collaborative filtering algorithm.
\end{abstract}

% Each section begins with a \section{title} command
\section{Introduction}
    % \PARstart{}{} creates a tall first letter for this first paragraph
    \PARstart{A}{}recommender system  is a subclass of information filtering system that seeks to predict the "rating" or "preference" that a user would give to an item. Recommender systems have become increasingly popular in recent years, and are utilized in a variety of areas including movies, music, news, books, research articles, search queries, social tags, and products in general. In this experiment, we use matrix factorization algorithm to build a recommender system. 

% Main Part
\section{Methods and Theory}

    Matrix Factorization is the most widely used model-based collaborative filtering algorithm.

    Given a rating matrix $R$ of size $m \times n$, with sparse ratings from m users to n items. We assume matrix $R$ can be factorized into the multiplication of two low-rank feature matrices $P$ of size $m \times K$ and $Q$ of size $K \times n$. 

    To solve this, we define following objective function:
    \begin{equation}
        L=(r_{u,i}-p_u^Tq_i)^2+\lambda_p\|p_u\|^2+\lambda_q\|q_i\|^2
    \end{equation}

    We use SGD to optimize this object function. We also need to calculate prediction error:
    \begin{equation}
        E_{u,i}=r_{u,i}-p_u^Tq_i
    \end{equation}

    And gradient:
    \begin{equation}
        \begin{aligned}
            p_u&=p_u+\alpha(E_{u,i}q_i-\lambda_pp_u) \\
            q_i&=q_i+\alpha(E_{u,i}p_u-\lambda_qq_i)
        \end{aligned}
    \end{equation}

    The whole SGD algorithm is following:
    \begin{algorithm}[H]
        \caption{Matrix Factorization SGD Algorithm}
        \begin{algorithmic}[1]

            \STATE \textbf{Require} Feature matrices \textbf{P}, \textbf{Q},  observed set $\Omega$, regularization parameters $\lambda_p$, $\lambda_q$ and learning rate $\alpha$.
            \STATE \textbf{Randomly} select an observed sample $r_{u,i}$ from observed set $\Omega$.
            \STATE Calculate the \textbf{gradient} w.r.t to the objective function: 
            $$
            \begin{aligned}
                E_{u,i}&=r_{u,i}-p_u^Tq_i \\
                \frac {\partial L}{\partial p_u}&=E_{u,i}(-q_i)+\lambda_pp_u \\
                \frac {\partial L}{\partial q_i}&=E_{u,i}(-p_u)+\lambda_qq_i
            \end{aligned}
            $$
            \STATE \textbf{Update} the feature matrices \textbf{P} and \textbf{Q} with learning rate $\alpha$ and gradient:
            $$p_u=p_u+\alpha(E_{u,i}q_i-\lambda_pp_u)$$
            $$q_i=q_i+\alpha(E_{u,i}p_u-\lambda_qq_i)$$
            \STATE \textbf{Repeat} the above processes until \textbf{convergence}.
        \end{algorithmic} 
    \end{algorithm}

\section{Experiments}
\subsection{Dataset}
    We use MovieLens-100k dataset which consists 10,000 comments from 943 users out of 1682 movies. At least, each user comment 20 videos. Users and movies are numbered consecutively from number 1 respectively. The data is sorted randomly.
    In this dataset, u1.base / u1.test are train set and validation set respectively, seperated from the dataset with proportion of 80\% and 20\%

\subsection{Implementation}

    There are many algorithms to solve this problem. We use stochastic gradient descent(SGD). The steps is following:
    
    \begin{enumerate}[1.]
        \item Read the dataset. We use u1.base / u1.test directly. Populate the original scoring matrix $R_{n\_users,n\_items}$ against the raw data, and fill 0 for null values.
        \item Initialize the user factor matrix $P_{n\_users,K}$ and the item (movie) factor matrix $Q_{n\_items,K}$, where $K$ is the number of potential features.
        \item Determine the loss function and hyperparameter learning rate $\alpha$ and the penalty factor $\lambda$.
        \item Use the stochastic gradient descent method to decompose the sparse user score matrix, get the user factor matrix and item (movie) factor matrix:

        \begin{enumerate}[(1)]
            \item Select a sample from scoring matrix randomly;
            \item Calculate this sample's loss gradient of specific row(column) of user factor matrix and item factor matrix;
            \item Use SGD to update the specific row(column) of $P_{n\_users,K}$ and $Q_{n\_items,K}$;
            \item Calculate the $L_{validation}$ on the validation set, comparing with the $L_{validation}$ of the previous iteration to determine if it has converged.
        \end{enumerate}
        \item Repeat step 4. several times, get a satisfactory user factor matrix $P$ and an item factor matrix $Q$, Draw a $L_{validation}$ curve with varying iterations.
        \item The final score prediction matrix $\hat{R}_{n\_users,n\_items}$ is obtained by multiplying the user factor matrix $P_{n\_users,K}$ and the transpose of the item factor matrix $Q_{n\_items,K}$.
    \end{enumerate}
    
\subsection{Result}
    We did a lot of experiments and found that the model would overfit when the regularization term was small.
    We found parameters that would fit the model, as Table.~\ref{tab:parameters}
	

	% This is how you define a table: the [!hbt] means that LaTeX is forced (by the !) to place the table exactly here (by h), or if that doesnt work because of a pagebreak or so, it tries to place the table to the bottom of the page (by b) or the top (by t).
	\begin{table}[!hbt]
		% Center the table
		\begin{center}
		% Title of the table
		\caption{Parameters}
		\label{tab:parameters}
		% Table itself: here we have two columns which are centered and have lines to the left, right and in the middle: |c|c|
		\begin{tabular}{|c|c|}
			% To create a horizontal line, type \hline
			\hline
			% To end a column type &
			% For a linebreak type \\
			learning rate & 0.1 \\
			\hline
			$\lambda_p$ & 10 \\
            \hline
            $\lambda_q$ & 10 \\
			\hline
		\end{tabular}
		\end{center}
	\end{table}


    We plot the loss on validation set as Fig.~\ref{fig:loss}. We can see that as the number of iterations increases, loss decreases. This tell us SGD is a useful optimize algorithm for matrix factorization
	% If you have questions about how to write mathematical formulas in LaTeX, please read a LaTeX book or the 'Not So Short Introduction to LaTeX': tobi.oetiker.ch/lshort/lshort.pdf

	% This is how you include a eps figure in your document. LaTeX only accepts EPS or TIFF files.
	\begin{figure}[!hbt]
		% Center the figure.
		\begin{center}
		% Include the eps file, scale it such that it's width equals the column width. You can also put width=8cm for example...
		\includegraphics[width=\columnwidth]{loss}
		% Create a subtitle for the figure.
		\caption{SGD algorithm loss on validation set.}
		% Define the label of the figure. It's good to use 'fig:title', so you know that the label belongs to a figure.
		\label{fig:loss}
		\end{center}
	\end{figure}


\section{Conclusion}
    In this experiment, we implemented a recommendation system based on matrix factorization. This gives is a deeper understanding of the principles of recommendation systems. At the same time, we also consolidated the knowledge of matrix factorization and stochastic gradient descent algorithm.


% Your document ends here!
\end{document}